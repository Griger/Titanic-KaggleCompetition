\documentclass[10pt,a4paper]{article}
\usepackage[utf8]{inputenc}
\usepackage{amsmath}
\usepackage{amsfonts}
\usepackage{amssymb}

\usepackage{float}
\usepackage[table,xcdraw]{xcolor} %para usar tablas con color de fondo en las celdas
\usepackage{hyperref} %para poder poner enlaces
\usepackage{listings} %para insertar código
\usepackage{tikz}%para pintar las redes neuronales
\usepackage{color} %para poder definir y usar colores
\usepackage{soul} %para hacer los subrayados

\author{\textbf{Gustavo Rivas Gervilla}}
\title{\textcolor{deepblue}{\textbf{Titanic. Competición Kaggle.}}}
\date{}

%Configurando lstlisting para mostrar código Python con algún 	 de colores (copiado de http://tex.stackexchange.com/questions/83882/how-to-highlight-python-syntax-in-latex-listings-lstinputlistings-command) ------------------------------
% Custom colors
\definecolor{deepblue}{rgb}{0,0,0.5}
\definecolor{deepred}{rgb}{0.6,0,0}
\definecolor{deepgreen}{rgb}{0,0.5,0}
\definecolor{light-gray}{gray}{0.85}
\definecolor{comment-gray}{gray}{0.65}
\definecolor{light-green}{rgb}{0.66,1,0.5}
\definecolor{light-yellow}{rgb}{1,1,0.4}

% Default fixed font does not support bold face
\DeclareFixedFont{\ttb}{T1}{txtt}{bx}{n}{8} % for bold
\DeclareFixedFont{\ttm}{T1}{txtt}{m}{n}{8}  % for normal

%Configuración de los listings
\lstset{
	language=Python,
	basicstyle=\ttm,
	otherkeywords={self},             % Add keywords here
	keywordstyle=\ttb\color{deepblue},
	emph={MyClass,__init__},          % Custom highlighting
	emphstyle=\ttb\color{deepred},    % Custom highlighting style
	stringstyle=\color{deepgreen},
	frame=tb,                         % Any extra options here
	showstringspaces=false,            % 
	commentstyle=\ttm\color{comment-gray}, % Custom comment style
}
%--------------------------------------------------------------------------------

\newcommand{\emp}[1]{\sethlcolor{light-yellow}\hl{\texttt{#1}}} %Comando para poner código inline
\newcommand{\code}[1]{\sethlcolor{light-gray}\hl{\texttt{#1}}} %Comando para poner código inline
\newcommand{\archive}[1]{\sethlcolor{light-green}\hl{\texttt{#1}}} %Comando para resaltar nombres de archivos
\renewcommand\tablename{Tabla} %Cambiar el nombre de las tablas
\renewcommand\figurename{Figura} %Cambiar el nombre de las tablas
\renewcommand{\contentsname}{Índice} %Cambiar el nombre de la ToC

\begin{document}
\maketitle
\section{Tutorial ``Exploring Survival on the Titanic'' de Megan Risdal}

A modo de introducción tanto al dataset como a la plataforma Kaggle vamos a realizar este tutorial que nos servirá para dar una primera solución al problema que se nos plantea con este dataset. También, y simplemente por conocer esta herramienta, se ha optado por realizar este tutorial desarrollando un aplicación web Shiny. El fichero tiene extensión de Rmarkdown, algo que ya he usado, con lo cual no debería ser muy complicado usar esta utilidad que ofrece Rstudio.

\subsection{Preprocesamiento}

Lo primero que observamos es el uso de la función \code{bind\_rows} del paquete \code{dplyr}, lo que hacemos con esta función es unir los datasets de train y test en uno solo, la diferencia entre esta función y usar por ejemplo \code{rbind} es que nos facilita el trabajo, ya que mientras que \code{rbind} nos da un error al no tener ambos datasets el mismo número de columnas (el conjunto de test no contiene la clase a la que pertenece cada una de las instancias ya que es lo que hemos de predecir de cara a la competición), con \code{bind\_rows} lo que hace es unir ambos e insertar \textbf{NA} en aquellos atributos que no están presentes en el dataset de test, la clase de los pasajeros.\\

En el tutorial realiza una pequeña exploración sobre el conjunto total de instancias del problema haciendo uso de la función \code{str} aunque esto no nos aporta información acerca de la semántica de cada uno de los atributos del dataset, estos son los siguientes:

\begin{enumerate}
\item Survived: la clase a predecir. Sobrevivió (1) o murió (0).
\item Pclass: la clase del pasaje.
\item Name: el nombre del pasajero.
\item Sex: sexo del pasajero.
\item Age: edad del pasajero.
\item SibSp
\item Parch
\item Ticket
\item Fare: 
\item Cabin: número de camarote.
\item Embarked: puerto en el que embarcó.
\end{enumerate}

\subsection{Subiendo resultados a Kaggle}

A modo de ejercicio lo que he hecho ha sido generar simplemente una predicción completamente aleatoria, de este modo podremos testear rápidamente el sistema de subida a la plataforma Kaggle y además tener un punto de referencia más para comparar los distintos experimentos que realicemos. Con esta predicción hemos obtenido un score de \textbf{0.43541}. Observemos que hemos puesto el atributo \code{quote} de la función \code{write.csv} de modo que el nombre de las columnas del dataset se introduzcan sin ir entre comillas dobles, de otro modo Kaggle no aceptaría el submisión.

Tras el tutorial de Riscal pos \textbf{989} accuracy \textbf{0.80383}

Tras aplicar Xgboost al dataset de Riscal \textbf{0.77990}.

\end{document}